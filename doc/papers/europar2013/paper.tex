% to submit via https://www.easychair.org/account/signin.cgi?conf=europar2013
% max 12 pages
\documentclass{llncs}
%
\begin{document}
%
\title{Pythran: Bringing OpenMP to a Python Subset}

\author{Serge Guelton\inst{1,2} \and Pierrick Brunet\inst{2}}

\institute{
    {\'E}cole Normale Sup{\'e}rieure, Paris, France
    \and
    T{\'e}l{\'e}com Bretagne, Plouzan{\'e}, France
}

\maketitle

%
\begin{abstract}

    The Python language offers an interesting alternative as a scientific
    computing language thanks to the SciPy package. However, the lack of
    fine-grained parallelism support remains an important shortcoming. This
    paper proposes to add OpenMP directives to the Python language and shows
    that it is possible to turn a module written in a large Python subset
    enhanced with these directive into a native parallel module that runs as
    fast as its naive C equivalent. The approach is validated through the
    Pythran compiler.

\keywords{Python, OpenMP, C++, static compilation, scripting language}
\end{abstract}

%
\section{Introduction}

Since its birth in December 1989, the Python language~\cite{rossum97} has proved
to be useful in various domains, ranging from system administration to web
services, thanks to its dynamicity, expressiveness, rich ecosystem and
battery-included standard library. It is also getting widely used in scientific
computing~\cite{Oliphant2007}, mainly thanks to the SciPy~\cite{scipy} project
that adds Matlab-like functionalities to the language.

The prohibitive performance overhead of the language implied by its interpreted
and highly dynamic nature should prevent its usage as a language for high
performance computing. SciPy overcomes this issue through the usage of low-level
routines written in C or Fortran and encapsulated in Python through
\emph{native} modules. Moreover, its core data structure, the multi-dimensional
array~\cite{numpyarray2011}, was designed so that the underlying data are
available to both native modules and the Python interpreter without conversion
costs.

To make it easier for the user to write native functions, \emph{i.e.}\ without
having to write the glue code that turns Python object into native structures
back and forth, the SciPy package provides the \texttt{weave} module that makes
it possible to bundle C code snippet into Python code and compile then load them
at runtime.

This hybrid approach, based on a mix of interpreted and native code, is getting
wide spread in the Python landscape. Section~\ref{sec:python-parallel} studies the
existing approaches and shows a critical lack of parallelism support.
Section~\ref{sec:python-openmp} studies the validity of adding OpenMP directives to the
Python language in order to add fine-grain parallelism support to Python in the
context of a Python subset called Pythran. Static compilation of this language
into C++ and the underlying runtime is described in
Section~\ref{sec:python-static-translation}.
Section~\ref{sec:python-optim} presents several optimizations applied by the compiler to
optimized generated code with respect to parallelism. The Pythran compiler is
benchmarked on several scientific applications and compared to both interpreted
and native code in Section~\ref{sec:validation}.

%=========================================================
\section{Parallel Computations in Python}\label{sec:python-parallel}
%=========================================================

In the many cores area, turning to parallelism to balance the performance
limitations of scripting languages, as described in~\cite{choy05} in the context
of the MATLAB language, or in~\cite{mals07} in the context of the R language,
seems legitimate. In the context of Python, most approaches have
focused on fork-based parallelism.

\subsection{Python and Parallelism}

Parallel computations are supported by the Python standard library through the
\texttt{multiprocessing} module. It spawns several interpreters that can
communicate through IPC, using Python built-in object serialization. This
approach is only viable for computation intensive application, but the
communication and synchronization overheads are much greater than a light-weight
processes approach.

Although the standard \texttt{threading} module makes it possible to start
several light-weight threads within the same interpreter, this approach is not
relevant for HPC, because of a specificity of CPython, the \emph{Global
Interpreter Lock}~\cite{gil2012}~\footnote{Other Python interpreters, such as
IronPython or Jython, do not have a GIL.} This lock ensures only one thread is
active at a time in the interpreter. While it make it possible to have
cooperative thread, say for a GUI, it does not take advantage of multiple cores.
However, there are two notable exceptions: the GIL is released on I/O, and the
GIl does not prevent the use of threads inside native modules, where the user
has full control.

As a consequence, Python developers need to write multi-threaded native
modules in order to fully benefit from multiple cores. This lead to a kind of
computations referred as hybrid computations.

%There have been several approaches to replace the GIL by Transactional
%Memories~\cite{Riley2006,Tabba2010} but none of them made there way to the
%mainstream interpreters.


\subsection{Hybrid Computations}

In the context of interpreted language, an hybrid computation is defined
in~\cite{dongara2007} as a computation where part of the code is interpreted,
and part of the code is executed natively.

It is now common for scripting language to have C bindings. To take advantage of
compiled code, and to overcome the GIL limitations, Python developers must write
parallel C/C++ functions and the associated boilerplate based on the Python C
API~\cite{pythoncapi}. Tools have been developed to relieve the user from
writing the encapsulating code, notably Swig~\cite{swig2003} that relies on an
enhanced interface specification, or
\texttt{boost::python}~\cite{boostpython2007} that relies on C++ facilities to
guide translation.

An opposite approach consists in using the host language ---herein Python---
to describe both parts of the system, and to let an automated tool perform the
translation to native code of a part of the application, generally the
computation-intensive one where parallelism has been expressed in some ways.
Thus, developers not familiar with lower level language or not eager to invest
the additional development time can still benefit from a fair performance boost.
This approach has been subject to many studies that can be classified based on
their backward-compatibity with the host language.

\subsubsection{Backward-Incompatible Approaches}

A constraining (from the performance point of view) aspect of the Python
language is its type system. It has several consequences, among which the fact
that each method call is resolved dynamically, including a simple add
operation! It comes at no surprise that many approaches extend the Python
language to add a static typing overlay. Likewise, there is only two types of
integers (64bits integers and multi-precision integers) and one type of floating
point type (double-precision floats) in Python, while using a type with the
appropriate size may lead to significant performance boost.

Cython~\cite{cython2010} is such a Python dialect. It extends the syntax with
typing informations, calls to native functions from third party libraries, and a
limited set of parallelism constructs, among which the possibility to define
parallel loops, but no task parallelism. Plw~\cite{dongara2007} proposes a
similar approach that mixes Python with C, using raw strings to hold the C code
and the parallel directives, that are limited to parallel for loops.  The
numba~\footnote{\emph{cf}.\ \texttt{https://github.com/numba/numba}} compiler
uses additional type information to generate sequential LLVM bytecode.  PyCUDA
and PyOpenCL~\cite{klockner2012} also mixes python with kernels embeded as raw
strings to target accelerators.

The core issue of these approaches is that they imply to modify, in a more or
less trivial way, the original code. They require to invest in a new language,
and the long-term preservation of this investment is not ensured.

\subsubsection{Backward-Compatible Approaches}

Most backward-compatible approaches also require to modify the input program,
but they do not extend the Python language, rather restrict it. As a
consequence, they remain compatible with the original language and do not suffer
from the drawbacks of the previous approach.

Copperhead~\cite{shedskin2006} is a functional, data parallel language embedded in
Python. It uses n-uplets, Numpy arrays and lists as its core data structure and
prohibits usage of many control-flow operators such as loops, enforcing the use
of the \texttt{map}, \texttt{filter} or \texttt{reduce} intrinsics to exhibit
parallelism. In returns, it can be efficiently compiled to either CUDA or C++
with calls to the Thrust~\footnote{\emph{cf}.\
\texttt{http://thrust.github.com/}} library. Python decorators are used to
identify hot-spots that are dynamically compiled to native code.

Tools such as PyPy~\cite{pypy2009}, a Python interpreter with a tracing JIT, or
Shedskin~\cite{shedskin2006}, a Python to C++ compiler also are viable way to
turn regular Python codes into optimized ones, but they do not offer support for
fine-grained parallelism beyond what the standard library
proposes.~\footnote{Support for STM in PyPy is a work in
progress.}~\footnote{Shedskin is compatible with several Python modules that
provide coarse-grained, fork-based parallelism.}

This article proposes to combine OpenMP~\cite{openmp3.1} parallel annotations
with a Python subset called Pythran to bring fine-grain parallelism to Python
while being backward compatible with both the Python language, and the
sequential algorithm. It means that:
\begin{enumerate}
    \item Any Pythran code can be run (sequentially), with no module dependency or code change,
        by any Python interpreter.
    \item Parallelism is explicit and incrementally added to the original code
        through directives.
\end{enumerate}

%=========================================================
\section{OpenMP Semantic Adaptations}\label{sec:python-openmp}
%=========================================================

\subsection{Variable Scope}

\subsection{Iterators}

%=========================================================
\section{Static Translation of Python Programs}\label{sec:python-static}
%=========================================================

\subsection{Directive Oblivious Translation}

\subsection{Runtime Support}

\subsubsection{Shared References}

\subsubsection{Iterator Category}

%=========================================================
\section{Optimizations for Parallelism}\label{sec:python-optim}
%=========================================================

\subsection{Limiting Copies}

\subsection{Transfer Costs}

%=========================================================
\section{Validation}\label{sec:validation}
%=========================================================

\subsection{Comparison with CPython}

\subsection{Comparison with Cython}

%=========================================================
\section{Conclusion and Future Work}
%=========================================================

\bibliographystyle{splncs03}
\bibliography{biblio}

\end{document}
% vim:spell spelllang=en
% vim:tw=80
